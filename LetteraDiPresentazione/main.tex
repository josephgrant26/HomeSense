\documentclass[11pt,a4paper]{article}
\usepackage[utf8]{inputenc}
\usepackage[italian]{babel}
\usepackage{geometry}
\usepackage{graphicx} % Required for inserting images
\usepackage{float}
\usepackage{tabularx}
\usepackage{booktabs}
\usepackage{hyperref}
\usepackage{fancyhdr}
\usepackage{lastpage}

\geometry{margin=2.5cm}

% Intestazione e piè di pagina
\pagestyle{fancy}
\fancyhf{}
\renewcommand{\headrulewidth}{0pt}
\fancyfoot[C]{\thepage\ di \pageref{LastPage}}

\begin{document}

% Prima pagina - Intestazione
\begin{titlepage}
    \centering
    
    % Logo Università
    \begin{figure}[H]
        \centering
        \includegraphics[width=0.3\textwidth]{logoUnipd.jpg}
    \end{figure}
    
    \vspace{0.5cm}
    
    \textbf{\Large Università degli Studi di Padova} \\
    \vspace{0.2cm}
    \textbf{Laurea in Informatica} \\
    \vspace{0.2cm}
    \textbf{Corso di Ingegneria del Software} \\
    \vspace{0.2cm}
    \textbf{Anno Accademico 2025/2026} \\
    
    \vspace{1cm}
    
    % Logo gruppo
    \begin{figure}[H]
        \centering
        \includegraphics[width=0.2\textwidth]{LogoByteMe.jpg}
    \end{figure}
    
    \vspace{0.5cm}
    
    \textbf{\Large Gruppo ByteMe} \\
    \vspace{0.2cm}
    \texttt{byteme2025swe@gmail.com} \\
    
    \vspace{1cm}
    
    \textbf{\Huge Lettera di Presentazione} \\
    
    \vspace{1.5cm}
    
    \begin{table}[H]
        \centering
        \begin{tabular}{|l|p{5cm}|}
            \hline
            \textbf{Versione} & 1.0.0 \\
            \hline
            \textbf{Stato} & Approvato \\
            \hline
            \textbf{Redazione} & Giulia Barzon \\ & Nome Cognome \\
            \hline
            \textbf{Verifica} & Nome Cognome \\
            \hline
            \textbf{Approvazione} & Tutto il gruppo \\
            \hline
            \textbf{Proprietario} & ByteMe \\
            \hline
            \textbf{Uso} & Esterno \\
            \hline
            \textbf{Destinatari} & Prof. Tullio Vardanega \\ & Prof. Riccardo Cardin \\
            \hline
        \end{tabular}
    \end{table}
    
    \vfill
\end{titlepage}

% Seconda pagina - Indice
\tableofcontents
\thispagestyle{empty}
\newpage

% Terza pagina - Contenuto principale
\setcounter{page}{1}

\section{Presentazione}
Alla cortese attenzione del Prof. Tullio Vardanega e del Prof. Riccardo Cardin.

Il gruppo ByteMe intende formalmente candidarsi per la realizzazione del capitolato C6 "Progetto Second Brain", proposto dall'azienda Zucchetti, confermando il proprio impegno per lo sviluppo del progetto.

Viene riportato di seguito il link alla repository dove è possibile consultare la documentazione del gruppo:
\begin{center}
\texttt{https://github.com/ByteMe25/ByteMe}
\end{center}

\noindent All'interno troverete:
\begin{itemize}
    \item Lettera di presentazione
    \item Valutazione dei capitolati
    \item Preventivo dei costi e impegno orario dei membri del gruppo
    \item Verbali delle riunioni del gruppo e dei meeting con le aziende
\end{itemize}

\newpage
\section{Resoconto dell'incontro con Zucchetti}
In data 21 ottobre 2025, il gruppo ByteMe ha partecipato ad un meeting online con il rappresentante dell'azienda Zucchetti. L'incontro, finalizzato a chiarire gli aspetti del progetto, ha permesso di ottenere risposte esaustive a tutti i quesiti del team.

Di seguito vengono riportate le domande sollevate e una sintesi delle risposte fornite.

\subsection{Modalità di comunicazione e supporto}
\textbf{Argomento:} Definizione dei canali di comunicazione preferiti, frequenza e modalità degli aggiornamenti di stato attesi, procedura per la gestione delle problematiche.

\textbf{Risposta:} L'azienda si è dimostrata flessibile, lasciando al gruppo la scelta della modalità di contatto. Ha proposto meeting online tramite Google Meet e la possibilità di incontri in presenza presso la propria sede. Zucchetti ha garantito piena disponibilità e supporto per la risoluzione di problematiche, richiedendo unicamente un preavviso congruo per organizzare al meglio il supporto. La pianificazione di incontri periodici (per consultazione, gestione delle criticità e review degli sprint) verrà definita formalmente in seguito all'assegnazione del capitolato.

\subsection{Linee Guida per l'Interfaccia Utente}
\textbf{Argomento:} Preferenze dell'azienda riguardanti lo stile dell'interfaccia.

\textbf{Risposta:} L'azienda non ha imposto vincoli specifici sullo stile dell'interfaccia, lasciando piena autonomia decisionale al gruppo. Come riferimento utile, è stata segnalata una libreria online che il team potrà liberamente consultare.

\subsection{Indicazioni sulla Tecnologia di Sviluppo}
\textbf{Argomento:} Sviluppo come applicazione o sito web tradizionale.

\textbf{Risposta:} Il progetto Second Brain è pensato come sito web. La natura del progetto 'Second Brain' è quella di una web application, concepita per essere fruita interamente attraverso un browser web standard, senza necessità di installazione locale.

\subsection{Approfondimenti sulle Funzionalità del Progetto}
\textbf{Argomento:} Approfondimenti sulle funzionalità del progetto, con particolare riferimento agli aspetti non pienamente definiti nel capitolato.

\textbf{Risposta:} L'azienda ha sottolineato come l'intelligenza artificiale rappresenti l'elemento cardine del progetto, illustrandone il ruolo attraverso esempi pratici e un confronto tra diversi modelli. Uno dei requisiti principali è l'implementazione di un sistema in cui l'utente possa fornire un input testuale breve, come un'idea o un pensiero, a partire dal quale l'AI genererà autonomamente un testo esteso e strutturato. A garanzia dello sviluppo, Zucchetti si impegna a fornire le chiavi API necessarie per l'integrazione e l'utilizzo dei servizi di AI selezionati.

\subsection{Sviluppo per dispositivi mobili}
\textbf{Argomento:} È richiesta un'interfaccia utente dedicata che rende il sito utilizzabile da smartphone oppure è sufficiente occuparsi solo dell'esperienza per desktop.

\textbf{Risposta:} La natura del progetto prevede interfacce complesse che richiedono ampio spazio visivo. Pertanto, implementare un'interfaccia dedicata per schermi di piccole dimensioni non è considerato un requisito obbligatorio. Tuttavia, l'esplorazione di un design responsive o mobile-friendly rimane un requisito opzionale che il gruppo può liberamente valutare.

\newpage
\section{Motivazioni della scelta}
Il gruppo ByteMe ha selezionato il capitolato C6 "Progetto Second Brain" dopo un'attenta valutazione, ritenendo che le sue caratteristiche si allineino perfettamente con le competenze che il team può offrire.

Le motivazioni alla base di questa scelta sono le seguenti:
\begin{itemize}
    \item \textbf{Interesse tecnologico}: L'utilizzo di tecnologie AI all'avanguardia rappresenta un'opportunità di crescita tecnica e di sperimentazione di nuovi strumenti per tutti i membri del team.
    \item \textbf{Fattibilità}: Il progetto rappresenta un buon compromesso tra complessità e possibilità di realizzazione nel tempo disponibile, permettendo al team di applicare le competenze web in acquisizione senza richiedere l'uso di tecnologie eccessivamente complesse.
    \item \textbf{Supporto aziendale}: Zucchetti ha dimostrato grande disponibilità al dialogo e al supporto tecnico e ha dato un'impressione più che positiva al gruppo.
    \item \textbf{Autonomia progettuale}: La libertà nelle scelte tecnologiche e implementative permette di valorizzare le competenze del team.
\end{itemize}

\newpage
\section{Conclusione}
A corredo della presente, il gruppo ha redatto il documento "Preventivo dei costi e impegni orari", che analizza nel dettaglio la suddivisione dei ruoli, l'impegno orario del team, i potenziali rischi e il preventivo di spesa.

\begin{table}[H]
    \centering
    \begin{tabular}{|c|c|c|}
        \hline
        \textbf{Cognome} & \textbf{Nome} & \textbf{Matricola} \\
        \hline
        Cuogo & Matteo & 2111013 \\
        \hline
        Grant & Joseph & 1224441 \\
        \hline
        Grossele & Chiara & 2101063 \\
        \hline
        Marchioro & Elisa & 2111941 \\
        \hline
        Barzon & Giulia & 2101074 \\
        \hline
        Tombacco & Tommaso & 2076447 \\
        \hline
    \end{tabular}
\end{table}

\noindent Confidando in una valutazione positiva della nostra proposta, restiamo a disposizione per ogni eventuale chiarimento e porgiamo distinti saluti.

\hfill Il gruppo ByteMe

\end{document}
